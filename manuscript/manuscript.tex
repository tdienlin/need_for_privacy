\documentclass[man]{apa6}
\usepackage{lmodern}
\usepackage{amssymb,amsmath}
\usepackage{ifxetex,ifluatex}
\usepackage{fixltx2e} % provides \textsubscript
\ifnum 0\ifxetex 1\fi\ifluatex 1\fi=0 % if pdftex
  \usepackage[T1]{fontenc}
  \usepackage[utf8]{inputenc}
\else % if luatex or xelatex
  \ifxetex
    \usepackage{mathspec}
  \else
    \usepackage{fontspec}
  \fi
  \defaultfontfeatures{Ligatures=TeX,Scale=MatchLowercase}
\fi
% use upquote if available, for straight quotes in verbatim environments
\IfFileExists{upquote.sty}{\usepackage{upquote}}{}
% use microtype if available
\IfFileExists{microtype.sty}{%
\usepackage{microtype}
\UseMicrotypeSet[protrusion]{basicmath} % disable protrusion for tt fonts
}{}
\usepackage{hyperref}
\hypersetup{unicode=true,
            pdftitle={Who Needs Privacy?},
            pdfauthor={Tobias Dienlin~\& Miriam Metzger},
            pdfkeywords={keywords},
            pdfborder={0 0 0},
            breaklinks=true}
\urlstyle{same}  % don't use monospace font for urls
\usepackage{graphicx,grffile}
\makeatletter
\def\maxwidth{\ifdim\Gin@nat@width>\linewidth\linewidth\else\Gin@nat@width\fi}
\def\maxheight{\ifdim\Gin@nat@height>\textheight\textheight\else\Gin@nat@height\fi}
\makeatother
% Scale images if necessary, so that they will not overflow the page
% margins by default, and it is still possible to overwrite the defaults
% using explicit options in \includegraphics[width, height, ...]{}
\setkeys{Gin}{width=\maxwidth,height=\maxheight,keepaspectratio}
\IfFileExists{parskip.sty}{%
\usepackage{parskip}
}{% else
\setlength{\parindent}{0pt}
\setlength{\parskip}{6pt plus 2pt minus 1pt}
}
\setlength{\emergencystretch}{3em}  % prevent overfull lines
\providecommand{\tightlist}{%
  \setlength{\itemsep}{0pt}\setlength{\parskip}{0pt}}
\setcounter{secnumdepth}{0}
% Redefines (sub)paragraphs to behave more like sections
\ifx\paragraph\undefined\else
\let\oldparagraph\paragraph
\renewcommand{\paragraph}[1]{\oldparagraph{#1}\mbox{}}
\fi
\ifx\subparagraph\undefined\else
\let\oldsubparagraph\subparagraph
\renewcommand{\subparagraph}[1]{\oldsubparagraph{#1}\mbox{}}
\fi

%%% Use protect on footnotes to avoid problems with footnotes in titles
\let\rmarkdownfootnote\footnote%
\def\footnote{\protect\rmarkdownfootnote}


  \title{Who Needs Privacy?}
    \author{Tobias Dienlin\textsuperscript{1}~\& Miriam Metzger\textsuperscript{1,2}}
    \date{}
  
\shorttitle{Who Needs Privacy?}
\affiliation{
\vspace{0.5cm}
\textsuperscript{1} University of Hohenheim\\\textsuperscript{2} University of California, Santa Barbara}
\keywords{keywords\newline\indent Word count: X}
\usepackage{csquotes}
\usepackage{upgreek}
\captionsetup{font=singlespacing,justification=justified}

\usepackage{longtable}
\usepackage{lscape}
\usepackage{multirow}
\usepackage{tabularx}
\usepackage[flushleft]{threeparttable}
\usepackage{threeparttablex}

\newenvironment{lltable}{\begin{landscape}\begin{center}\begin{ThreePartTable}}{\end{ThreePartTable}\end{center}\end{landscape}}

\makeatletter
\newcommand\LastLTentrywidth{1em}
\newlength\longtablewidth
\setlength{\longtablewidth}{1in}
\newcommand{\getlongtablewidth}{\begingroup \ifcsname LT@\roman{LT@tables}\endcsname \global\longtablewidth=0pt \renewcommand{\LT@entry}[2]{\global\advance\longtablewidth by ##2\relax\gdef\LastLTentrywidth{##2}}\@nameuse{LT@\roman{LT@tables}} \fi \endgroup}


\DeclareDelayedFloatFlavor{ThreePartTable}{table}
\DeclareDelayedFloatFlavor{lltable}{table}
\DeclareDelayedFloatFlavor*{longtable}{table}
\makeatletter
\renewcommand{\efloat@iwrite}[1]{\immediate\expandafter\protected@write\csname efloat@post#1\endcsname{}}
\makeatother
\usepackage{lineno}

\linenumbers
\setlength{\parskip}{0em}
\raggedbottom

\authornote{Add complete departmental affiliations for each
author here. Each new line herein must be indented, like this line.

Enter author note here.

Correspondence concerning this article should be addressed to Tobias
Dienlin, University of Hohenheim, Department of Media Psychology (540F).
E-mail:
\href{mailto:tobias.dienlin@uni-hohenheim.de}{\nolinkurl{tobias.dienlin@uni-hohenheim.de}}}

\abstract{
Objective: This study analyzes how personality relates to peoples'
desire for privacy. Specifically, we investigated whether the syllogism
``I don't mind surveillance because I have nothing to hide'' is correct:
Do people who lack integrity (given they have something to hide) indeed
desire more privacy?

Method: Study 1 featured an online questionnaire (\emph{N} = 268,
\emph{M}\textsubscript{age} = 20 years, 72\% female) and Study 2 a
laboratory experiment (\emph{N} = 87, \emph{M}\textsubscript{age} = 20
years, 51\% female), where participants wrote an essay about past
negative, positive, or neutral behaviors to analyze effects on desire
for privacy.

Results: Study 1 showed that respondents who are more shy, less anxious,
and more risk averse desired more privacy. Respondents who self-reported
lacking integrity reported desiring more privacy from government and
more anonymity. Study 2 replicated these results and showed a
statistical trend (\emph{p} = .052) that writing about negative past
behaviors increased desire for interpersonal privacy. Moreover, the
integrity IAT showed significant relations with desire for privacy from
government.

Conclusion: It is possible to predict peoples' desire for privacy based
on their lack of integrity. However, other neutral personality facets
also explain desire for privacy. Hence, putting everyone who desires
privacy under general suspicion would be incorrect.


}

\usepackage{amsthm}
\newtheorem{theorem}{Theorem}[section]
\newtheorem{lemma}{Lemma}[section]
\theoremstyle{definition}
\newtheorem{definition}{Definition}[section]
\newtheorem{corollary}{Corollary}[section]
\newtheorem{proposition}{Proposition}[section]
\theoremstyle{definition}
\newtheorem{example}{Example}[section]
\theoremstyle{definition}
\newtheorem{exercise}{Exercise}[section]
\theoremstyle{remark}
\newtheorem*{remark}{Remark}
\newtheorem*{solution}{Solution}
\begin{document}
\maketitle

In his novel \emph{The Circle}, ({\textbf{???}}) describes a dystopian
society in which people are gradually forfeiting their privacy. People
decide to become \enquote{transparent}, which means that they start
carrying a small camera around the neck in order to broadcast their
daily lives to the Internet. Eventually, this causes a societal
upheaval: \enquote{The pressure on those who hadn't gone transparent
went from polite to oppressive. The question, from pundits and
constituents, was obvious and loud: If you aren't transparent, what are
you hiding?} ({\textbf{???}}). The main argument being offered to
justify the surveillance is: \enquote{If you have nothing to hide, you
have nothing to fear.} This syllogism is familiar, given that it
commonly appears in also nonfictional conversations ({\textbf{???}}).
Consider, for example, the following tweet: \enquote{I don't download
illegally. I don't have anything on my comp{[}uter{]} to hide. Hell, I'm
sure the \#NSA gave up on me years ago.} ({\textbf{???}}).

Why do people desire privacy? To date there is only little research on
why people desire privacy and how the desire for privacy can be
predicted by aspects of personality. For example, and to the best of our
knowledge, so far no study exists that has analyzed the nothing-to-hide
argument from a scientific and empirical perspective. Why do some people
not care whether government agencies such as the NSA are collecting
their data ({\textbf{???}}), and why do others protest vehemently in
order to protect their privacy?

Answering this question is important: Given that government agencies are
collecting large amounts of data hoping to reduce criminality and
terrorism, and given that government agencies are collecting this data
preemptively and without concrete suspicions, it is relevant to find out
whether this practice of mass surveillance can be justified based on the
nothing-to-hide argument. As a result, the main question of this paper
is: Do people who desire more privacy really have more to hide and, more
generally, what are personality facets that determine peoples' overall
desire for privacy?

\hypertarget{desire-for-privacy}{%
\subsection{Desire for Privacy}\label{desire-for-privacy}}

Privacy captures the extent of voluntary withdrawal from others (Westin,
1967). Several models suggest that privacy is a multi-dimensional
concept: For example, in a theory-driven treatise ({\textbf{???}})
argued that privacy has four dimensions: informational, social,
psychological, and physical privacy. Pedersen (1979), by contrast, did
an empirical factor analysis (initially starting with 94 items) and
suggested that privacy exists on six dimensions: reserve, isolation,
solitude, intimacy with friends, intimacy with family, and anonymity. In
addition, Schwartz (1968) differentiated between horizontal and vertical
privacy: Whereas horizontal privacy captures withdrawal from peers,
vertical privacy refers to withdrawal from superiors or institutions
(e.g., government agencies). Next to being multi-dimensional, privacy is
also contingent (Dienlin, 2014): One can, for example, distinguish
between the objective privacy context, the subsequent subjective
perception of privacy, the psychological desire for privacy (which is
both a situational and dispositional need), and the resulting privacy
behavior (as represented by self-disclosure). For the purpose of this
study, we combine the aforementioned theories and focus on (a) vertical
privacy with regard to the desire for withdrawal from government
surveillance, (b) horizontal privacy in terms of the desire for
withdrawal from peers, friends, or acquaintances, and (c) both
horizontal and vertical privacy as captured by the general desire for
anonymity.

\hypertarget{the-relation-between-integrity-and-desire-for-privacy}{%
\subsubsection{The relation between integrity and desire for
privacy}\label{the-relation-between-integrity-and-desire-for-privacy}}

Which specific aspects of personality help predict desire for privacy?
At its core, the nothing-to-hide argument implies that lack of integrity
is an important predictor of why people desire privacy. This becomes
especially apparent when we consider the definition of Solove's (2007)
nothing-to-hide argument (notably, Solove is a strong critic of the
nothing-to-hide argument):

The NSA surveillance, data mining, or other government information
gathering programs will result in the disclosure of particular pieces of
information to a few government officials, or perhaps only to government
computers. This very limited disclosure of the particular information
involved is not likely to be threatening to the privacy of law-abiding
citizens. Only those who are engaged in illegal activities have a reason
to hide this information. {[}({\textbf{???}}); p.~753{]}

This definition helps illustrate the link between lack of integrity and
desire for privacy: People who have \enquote{engaged in illegal
activities} can be considered, by definition, to lack integrity
({\textbf{???}}), which is why they have a reason \enquote{to hide this
information} (or, in other words, to desire more privacy). In terms of a
scientific definition of integrity there is no real consensus, however
most scholars agree that integrity \enquote{incorporates a tendency to
comply with social norms, avoid deviant behavior, and embrace a sense of
justice, truthfulness, and fairness} {[}({\textbf{???}}); p.~82{]}.

Several theoretical arguments exist why lack of integrity might
correlate with desire for privacy. In general, any self-disclosure is a
potential risk because others might disagree, disapprove, or misuse the
information in other contexts ({\textbf{???}}). Privacy regulation
theory showed that if self-disclosures are too risky, people raise their
desired level of privacy, intensify their boundary regulation, and
employ more mechanisms to seclude and protect themselves
({\textbf{???}}). In traditional contexts, this could range from
moderate behaviors like closing doors, to extreme behaviors such as
physically tossing someone out of the room ({\textbf{???}}). In modern
contexts, protecting one's privacy can mean to avoid photographs or to
deliberately shun public places that have surveillance cameras. People
who have actually committed something bad, treacherous, or illegal
become even more vulnerable and face a significant risk of
self-disclosure, because others will surely disapprove of these
activities ({\textbf{???}}). Hence, the foregoing arguments illuminate
an indirect link between integrity and desire for privacy: By
definition, people who participate in negative activities are considered
to lack integrity ({\textbf{???}}). People who have engaged in negative
activities have, by definition, more to hide, and disclosures concerning
those activities pose a high risk. Because of this increased risk,
people will arguably desire more privacy, as a means to mitigate their
felt risk ({\textbf{???}}). In this way, the current research extends
Altman's privacy regulation theory (1976) by suggesting that lack of
integrity is an important yet unexamined factor that could increase
peoples' desired level of privacy.

A few studies can be found that imply a relation between privacy and
integrity. For example, several studies found that surveillance reduces
cheating behaviors ({\textbf{???}}, Covey.1989). ({\textbf{???}}) asked
students to solve an impossible maze. In the high surveillance
condition, the experimenter stood in front of the students and closely
monitored their behavior. In the low surveillance condition, the
experimenter stood behind the students, did not monitor their behavior,
and visual dividers were used to block the experimenter's view of the
students. Results showed that students were more likely to cheat in the
low surveillance condition, suggesting that in situations of
surveillance (i.e., less privacy), people show fewer cheating behaviors
(i.e., more integrity). Similarly, people are more likely to prevent
others from stealing when security cameras are visible ({\textbf{???}}),
which is also a sign of higher integrity. Next, in a longitudinal sample
with 457 respondents in Germany ({\textbf{???}}), people who reported
needing more privacy were less satisfied with their lives (\emph{r} =
-.47), had more (\emph{r} = .41) and less positive affect (\emph{r} =
-.39). More importantly however, people who felt they needed more
privacy were also less authentic on their SNSs profiles (\emph{r} =
-.48) and less authentic in their personal relationships (\emph{r} =
-.28; {\textbf{???}}). For example, people who agreed to items like
\enquote{I do not talk about personal issues unless my conversation
partner brings them up first} were more likely to report that their
online profiles did not truly represent their personality. Given the
argument that authenticity is a subset of integrity ({\textbf{???}}), we
reason that the concept of integrity might relate to the desired level
of privacy. Finally, Pedersen (1982) showed that three dimensions of
need for privacy related to self-esteem: In his study with \emph{N} = 70
undergraduate students, respondents who held a lower self-esteem were
more reserved (\emph{r} = .29), needed more anonymity (\emph{r} = .21)
and preferred solitude (\emph{r} = .24). Granted, self-esteem and
integrity are generally distinct concepts; however, Pedersen's specific
operationalization of self-esteem integrated several aspects of
integrity (e.g., by using items such as \enquote{moral, nice, fair,
unselfish, good, honest, reputable, sane} to measure self-esteem). Thus,
our overarching hypothesis is that people who lack integrity have a
greater desire for privacy.

In Study 1, we used a questionnaire-based design to analyze how lack of
integrity and other personality facets relate to desire for privacy. In
accordance with the reasoning mentioned above, we suggest that people
with less integrity feel a greater desire for privacy. Specifically, we
argue that integrity may relate to the desire for privacy from (a)
government surveillance, as governments have the legitimate power to
prosecute illegal activities. Next, we hypothesize that integrity
relates to the desire privacy for (b) anonymity. Anonymity makes it more
difficult for both legal and social agents to identify and address
potential wrongdoers, which is why people with less integrity will
prefer situations in which they are anonymous. Finally, lack of
integrity likely also relates to an increased desire for privacy from
(c) other people, as most other people will disapprove of immoral or
illegal activities, and might reveal those activities to authorities.

Hypothesis 1: People who feel lower in self-perceived integrity desire
more privacy from government surveillance (H1a), more anonymity (H1b),
and more privacy from other persons (H1c).

\hypertarget{shyness}{%
\subsubsection{Shyness}\label{shyness}}

Critics of the nothing-to-hide argument hold that people who desire
privacy should not automatically be confronted with suspicion, and that
privacy has several purposes that are not related to criminal behavior
({\textbf{???}}). Westin (1967), for example, defined four primary
purposes of privacy: (1) self-development (i.e., the integration of
experiences into meaningful patterns), (2) autonomy (i.e., the desire to
avoid being manipulated and dominated), (3) emotional release (i.e., the
release of tension from social role demands), and (4) protected
communication (i.e., the ability to foster intimate relationships).
These are all important social factors for which people desire privacy.
Hence, the argument is that people who desire privacy can have several
legitimate reasons for doing so; reasons which are essential for
psychosocial wellbeing and which relate to different factors of
personality. Below, we thus explore other (neutral) aspects of
personality that potentially predict desire for privacy. In order to be
more precise, we follow the advice by Paunonen and Ashton (2001) and,
instead of using generic personality factors as predictors, refer to
specific personality facets.

First, we argue that people who are more reserved, who feel less
comfortable in social situations, generally desire more anonymity and
more interpersonal privacy. Given that privacy is, by definition, a
voluntary withdrawal from society (Westin, 1967), we expect that people
who are more reserved or more shy desire more privacy from others.
Several empirical studies support this hypothesis: Extroverted people
desire less privacy ({\textbf{???}}), people who describe themselves as
introverted thinkers are more likely to prefer social isolation
(Pedersen, 1982), and introverted people are more likely to report
invasions of privacy ({\textbf{???}}). Finally, we did not find
convincing theoretical and empirical arguments for why shyness should
relate to an increased desire for privacy from government surveillance,
which is why we did not include a hypothesis on this relation.

Hypothesis 2: People who are more shy desire more anonymity (H2a) and
more privacy from other persons (H2b).

\hypertarget{anxiety}{%
\subsubsection{Anxiety}\label{anxiety}}

Of course, there are also reasons why people might desire less privacy.
Government agencies often curtail privacy with the aim to prevent crime:
For example, the NSA's surveillance programs are often considered a
direct response to the 9 / 11 terrorists attacks ({\textbf{???}}). It
seems plausible that people who are more afraid of terrorist attacks are
also more likely to consent to these surveillance programs, given that
these programs promise to reduce the likelihood of future attacks. One
can then argue that people who are afraid of terrorist attacks are also
more afraid of threats overall, which is why we suggest that people who
are, in general, more anxious desire less privacy from government
surveillance and less anonymity. We did not include a hypothesis on the
potential relation between anxiety and desire for interpersonal privacy.
On the one hand, one could argue that people who are more anxious are
more reserved, given that social interactions can pose significant risks
(especially with strangers or weak ties; {\textbf{???}}). At the same
time, one could suggest that especially those people who are more
anxious desire less privacy from others (and especially their strong
ties), in order to cope better with their daily challenges. At the end,
given that we measure interpersonal privacy on a general level (and do
not distinguish between desire for privacy from (a) weak ties and (b)
strong ties), it seems plausible that both effects could cancel each
other out.

Hypothesis 3: People who are more anxious desire less privacy from
government surveillance (H3a) and more anonymity (H3b).

\hypertarget{risk-aversion}{%
\subsubsection{Risk aversion}\label{risk-aversion}}

Disclosing personal information always poses a certain risk, given that
others can misuse self-disclosed personal information in different
contexts, which can lead to severe consequences ({\textbf{???}}). Not
everyone will feel intimidated by this hypothetical threat---except
those who have a general tendency to avoid taking unnecessary risks. The
most cautious strategy to minimize risks of personal self-disclosures
would be, arguably, to keep as much information as possible private.
Hence, we suggest that people who are, in general, more risk averse have
a good reason to desire more privacy in all three aforementioned
contexts.

Hypothesis 4: People who are more risk averse desire more privacy from
government surveillance (H4a), more anonymity (H4b), and more privacy
from other persons (H4c).

\hypertarget{traditionality}{%
\subsubsection{Traditionality}\label{traditionality}}

The personal computer and the Internet have rendered the world
increasingly digitized: Social interactions, purchases, and medical
treatments nowadays all produce digital traces, which can be combined
into accurate latent user profiles. Given the features of digital
information (i.e., information is persistent, searchable, reproducible,
and scalable; {\textbf{???}}), this allows for unprecedented ways and
degrees of surveillance. Mark Zuckerberg famously observed that privacy
is no longer a \enquote{social norm,} rather that people share personal
information ({\textbf{???}}). Hence, in order to be part of contemporary
life (e.g., by using SNSs), it seems necessary to give up some privacy.
However, arguably not everyone is willing to pay that price, and
especially people who are more conservative might prefer to stick to
their usual routines and decide against giving up their privacy. This is
supported by empirical research: Older people, who are generally less
open and more traditional ({\textbf{???}}), are more concerned about
their privacy than younger people ({\textbf{???}}). Taken together, we
suggest that people who are more traditional also desire more privacy in
all three aforementioned contexts.

Hypothesis 5: People who are more traditional desire more privacy from
government surveillance (H5a), more anonymity (H5b), and more privacy
from other persons (H5c).

\hypertarget{methods}{%
\section{Methods}\label{methods}}

We report how we determined our sample size, all data exclusions (if
any), all manipulations, and all measures in the study.

\hypertarget{procedure-and-participants}{%
\subsection{Procedure and
participants}\label{procedure-and-participants}}

Participants were students from a university in the western U.S. who
received course credit for taking part in the study. The sample
consisted of \emph{N} = 296 respondents, with an age that ranged from 18
to 56 years (\emph{M} = 20 years). 72\% of the respondents were female.
The median participation time was 24 minutes. Regarding ethnicity, 37\%
of the respondents were Non-Hispanic White / Caucasian, 4\% Black /
African American, 21\% Hispanic / Latino, 24\% Asian / Pacific Islander,
0\% Native American, 5\% others, and 8\% nonresponse.

\hypertarget{measures}{%
\subsection{Measures}\label{measures}}

Despite the fact that we mostly used well-established scales,
confirmatory factor analyses (CFAs) showed that some of the original
items had to be deleted in order to achieve adequate factorial validity.
The final scales showed acceptable fit (CFI \textgreater .90, TLI
\textgreater .90, RMSEA \textless .10, SRMR \textless .10), good
composite reliability (REL(\(\upomega\)) \textgreater .60), and adequate
convergent factorial validity (AVE \textgreater .50; see Table
\ref{tab:DIS_Study_4_Table_1}). Respondents answered all items on a
7-point Likert scale ranging from 1 (\emph{strongly disagree}) to 7
(\emph{strongly agree}).

The data, all items (including deleted ones), results of CFAs, item
statistics, and distribution plots can be found in the online
supplementary material.\footnote{\url{https://osf.io/7ncpk/?view_only=38283fd9262646378e4ba1e19c9d707f}\}\}}

\hypertarget{desire-for-privacy-1}{%
\subsubsection{Desire for privacy}\label{desire-for-privacy-1}}

We measured desire for privacy on three dimensions: (a) Desire for
privacy from government surveillance, which represents the extent to
which people want the government to abstain from collecting information
about their personal life. One example item is \enquote{I feel the need
to protect my privacy from government agencies.} (b) Desire for
anonymity, which measures the extent to which people feel the need to
avoid identification(\enquote{I need to be able to use a fake name on
social network sites to preserve my privacy\}). (c) Desire for privacy
from other people, which measures the extent to which people want to
withhold personal information from others(}I don't feel the need to tell
my friends all my secrets\}). For each dimension, we used 3
self-developed items that build on prior studies ({\textbf{???}}).

\hypertarget{integrity}{%
\subsubsection{Integrity}\label{integrity}}

Integrity measures the extent to which people comply with social norms
and values. When measuring integrity, the question arises whether it is
possible to measure integrity based on self-reports. Interestingly,
integrity tests that are based on self-reports have been shown to work
successfully, given that they can predict unwanted professional
workplace behavior sufficiently (e.g., theft, drug and alcohol problems,
or absenteeism; {\textbf{???}}). In order to measure lack of integrity,
we thus used 4 items of the subscale integrity of the Supernumerary
Personality Inventory ({\textbf{???}}). An example item is \enquote{I
don't think there's anything wrong with cheating a little on one's
income tax forms.}

\hypertarget{shyness-1}{%
\subsubsection{Shyness}\label{shyness-1}}

Shyness captures whether people prefer to spend their time alone or in
company. We measured shyness with 4 items of the inverted extraversion
subscale gregariousness ({\textbf{???}}). An example item is "I shy away
from crowds of people.\}

\hypertarget{anxiety-1}{%
\subsubsection{Anxiety}\label{anxiety-1}}

Anxiety measures whether people are afraid of negative external
influences. We measured anxiety with 4 items of the neuroticism subscale
anxiety ({\textbf{???}}). An example item is "I am easily frightened.\}

\hypertarget{risk-avoidance}{%
\subsubsection{Risk avoidance}\label{risk-avoidance}}

Risk avoidance captures whether people abstain from taking risks. We
measured risk avoidance with 4 items of the conscientiousness subscale
deliberation ({\textbf{???}}). An example item is "I think twice before
I answer a question.\}

\hypertarget{traditionalism}{%
\subsubsection{Traditionalism}\label{traditionalism}}

Traditionalism measures whether people prefer to stick with their usual
routines. We measured traditionalism with 4 items of the inverted
openness to experiences subscale actions ({\textbf{???}}). An example
item is "I'm pretty set in my ways.\}

\hypertarget{data-analyses}{%
\subsection{Data analyses}\label{data-analyses}}

All hypotheses were tested with structural equation modeling (SEM). To
assess the SEM assumption of multivariate normality, we computed a
multivariate Shapiro-Wilk normality test. The results showed a violation
of multivariate normality (\emph{W} = 0.90, \emph{p} \textless .001),
which is why we used the more robust Satorra-Bentler scaled test
statistic as estimator. We treated missing data with casewise deletion
and tested all hypotheses with a two-tailed p \textless .050
significance level; values between \emph{p} = .050 and \emph{p} = .010
were considered trends toward significance. Regarding effect sizes, we
classified regression coefficients with values exceeding \(\upbeta\) =
.10 as small effects, \(\upbeta\) = .30 as medium effects, and
\(\upbeta\) = .50 as large effects.

Unfortunately, we did not determine sample size based on a priori power
analyses. We decided against including social desirability as a control
variable, because even though social desirability can affect answers to
sensitive questions ({\textbf{???}}), it is more likely to reflect a
true personality trait than false answering behavior ({\textbf{???}}).
Likewise, we did not include demographic control variables such as age
or education, because we used a typical student sample with little
demographic variance. We used R (Version 3.5.1; R Core Team, 2018) for
all our analyses.

\hypertarget{results}{%
\section{Results}\label{results}}

\hypertarget{discussion}{%
\section{Discussion}\label{discussion}}

The results of Study 1 showed that integrity relates to several
dimensions of desire for privacy: People who reported being of lower
integrity desired more privacy from government and more anonymity. In
other words, people who agreed that there would be nothing wrong with
cheating a little or lying occasionally were also more likely to agree
that the government should not invade peoples' privacy, even if that
could help to prevent terrorist attacks. Likewise, people who said, for
example, that they would feel tempted to take things that do not belong
to them were also more likely to avoid situations in which they were
identifiable.

In addition, desire for privacy was predicted also by other (neutral)
personality facets: People who were more shy, more risk averse, and less
anxious also desired more privacy. This implies that next to lack of
integrity there are various other personality-related aspects that
predict desire for privacy.

People who are more shy, more risk averse, or less anxious also desire
more privacy. For example, people who are less anxious are less likely
to accept government surveillance (arguably because they are less afraid
of terrorist attacks). When looking at the bigger implications of the
results, this shows the importance to make differentiated claims on why
people desire privacy: Indeed, the results suggest that some people
desire privacy because they might have something to hide. However,
putting everyone who desires privacy under a general suspicion is wrong
given that shy, risk averse, and less anxious people are also more
likely to desire privacy.

In conclusion, our results follow ({\textbf{???}}), who reasoned that if
exposure of information is risky it is likely that people will use more
mechanisms to strengthen their social boundaries and increase their
desired level of privacy. This study thus aligns with Altman's privacy
regulation theory by showing that, in several contexts, people with
lower integrity had a higher level of desired privacy.

\hypertarget{limitations-and-future-perspective}{%
\subsection{Limitations and future
perspective}\label{limitations-and-future-perspective}}

In our analysis of predictors of privacy, we followed the recommendation
by Paunonen and Ashton (2001) and did not analyze broad factors of
personality (e.g., neuroticism); instead, we focused on more specific
personality facets (e.g., fearfulness). For future research, we suggest
going one step further by analyzing predictors that are even more
specified. For example, it seems possible that people who hold
dissenting political beliefs could also have a higher desire for privacy
from the government. Similarly, it would be interesting to focus on
different minority groups. For example, it seems plausible that people
from a LGBT background might desire more privacy from government
(because it is potentially repressive or unfriendly toward LGBTs).
Finally, in this study we focused mostly on escapist motives for why
people desire privacy (e.g., shyness, risk aversion). Interestingly,
({\textbf{???}}) were able to show that when predicting engagement in
solitary activities, it is less preferable to measure how strongly
people want to escape society (avoidance oriented), but rather how much
they seek solitude (approach oriented). Hence, future studies might want
to include predictors that are more approach oriented (e.g., peoples'
desire for contemplation).

From a methodological perspective, future research should continue to
improve the instruments we used, given that factorial validity of some
scales was only moderate. Similarly, we recommend elaborating on the
general understanding of integrity as a theoretical concept. To date,
there is not one overarching concept of integrity that incorporates all
the different aspects of integrity, yet it would be valuable to examine
how other aspects of integrity (e.g., authenticity, trustworthiness, or
consistency) relate to privacy desires.

Power analyses showed that future research should use samples above
\emph{N} \(\approx\) 260 in order to test hypotheses with the
recommended power of at least .80 (Cohen, 1992).

In general, the question arises whether it is possible, or even socially
desirable, to measure a person's integrity. On the one hand, integrity
implies absolute criteria: Stealing is bad and forbidden, whereas
helping is good and encouraged. On the other hand, integrity implies
relative criteria: Whereas some cultures disapprove of lying whatever
the context, others consider lying okay---for example \enquote{white
lies} in order to save face or to avoid hurting someone's feelings
({\textbf{???}}). Thus, ranking behaviors, opinions, and character
traits with regard to integrity is a moral dilemma. As a result,
throughout the entire study we have understood integrity as a
transgression of social norms that is strong and that most societies
would agree upon (for example, most societies would consider stealing as
a sign of low integrity).

As a final note, we measured integrity based on self-ratings. One can
criticize this approach by saying it is not possible to measure
integrity based on self-reports because of social desirability
influences. However, self-reports of integrity can indeed predict
malevolent behavior: In a meta-analysis with 665 correlation
coefficients, integrity tests related to counterproductive behaviors
with a coefficient of \emph{r} = .47 ({\textbf{???}}). Nonetheless,
future research would benefit from including behavioral manifestations
of integrity, such as concrete cheating behaviors. If concrete cheating
behaviors also increase desires for privacy, this would strengthen the
underlying premise of the nothing-to-hide argument.

\hypertarget{conclusion}{%
\subsection{Conclusion}\label{conclusion}}

\newpage

\hypertarget{references}{%
\section{References}\label{references}}

\begingroup
\setlength{\parindent}{-0.5in}
\setlength{\leftskip}{0.5in}

\hypertarget{refs}{}
\leavevmode\hypertarget{ref-Dienlin.2014}{}%
Dienlin, T. (2014). The privacy process model. In S. Garnett, S. Halft,
M. Herz, \& J. M. Mönig (Eds.), \emph{Medien und Privatheit} (pp.
105--122). Passau, Germany: Karl Stutz.

\leavevmode\hypertarget{ref-Paunonen.2001}{}%
Paunonen, S. V., \& Ashton, M. C. (2001). Big Five factors and facets
and the prediction of behavior. \emph{Journal of Personality and Social
Psychology}, \emph{81}(3), 524--539.
doi:\href{https://doi.org/10.1037/0022-3514.81.3.524}{10.1037/0022-3514.81.3.524}

\leavevmode\hypertarget{ref-Pedersen.1979}{}%
Pedersen, D. M. (1979). Dimensions of privacy. \emph{Perceptual and
Motor Skills}, \emph{48}(3), 1291--1297.
doi:\href{https://doi.org/10.2466/pms.1979.48.3c.1291}{10.2466/pms.1979.48.3c.1291}

\leavevmode\hypertarget{ref-Pedersen.1982}{}%
Pedersen, D. M. (1982). Personality correlates of privacy. \emph{The
Journal of Psychology}, \emph{112}(1), 11--14.
doi:\href{https://doi.org/10.1080/00223980.1982.9923528}{10.1080/00223980.1982.9923528}

\leavevmode\hypertarget{ref-R-base}{}%
R Core Team. (2018). \emph{R: A language and environment for statistical
computing}. Vienna, Austria: R Foundation for Statistical Computing.
Retrieved from \url{https://www.R-project.org/}

\leavevmode\hypertarget{ref-Schwartz.1968}{}%
Schwartz, B. (1968). The social psychology of privacy. \emph{American
Journal of Sociology}, \emph{73}(6), 741--752.

\leavevmode\hypertarget{ref-Westin.1967}{}%
Westin, A. F. (1967). \emph{Privacy and freedom}. New York, NY:
Atheneum.

\endgroup


\end{document}
